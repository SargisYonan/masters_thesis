\chapter{Results}
The three path planners (HV, $N$-HV, and MCPP) introduced in Chapter \ref{ch:pp} all aim to reduce the overall prediction uncertainty of a field given a limited amount of flight time. They accomplish the task by calculating variances of a target field's predictions and attempting to choose a trajectory that reduces overall uncertainty. 

The number of trajectories compared in both the $N$-HV and MCPP methods is $N=5$. For the MCPP method, an additional $M_{mc}=10$ Monte Carlo trajectories are calculated for each of the $N$ trajectories. The target field size of the fields compared in the simulation have unit-less vesicle dimensions of $100\times 100$. Two random number generator seeds ($2$, $3$) are used to generate two sets of runs in an effort to show the methods for a variety of random fields. The autocorrelation factors of the field will be varied in an effort to show the effectiveness of the methods for different field statistics. When the prediction variances of the methods are compared, the values are normalized to an a priori mean variance, which is equal to the mean variance of the field generated from running a Kriging prediction on the equivalent field from a set of five samples taken from the main diagonal vesicles of the target field.

\subsection{Prediction Error Calculation}
The quality of each path planner will be judged by its ability to explore a field with a fixed exploration path length. The prediction error of each method will be used as a criterion of path planning quality.

The prediction error function, $E (Z,\hat{Z})$, will be the average root mean square (RMS) error for all $N\times N$ points on the actual field, $Z$, and the predicted field, $\hat{Z}$.

\begin{equation}
E (Z, \hat{Z}) = \frac{1}{N^2}\sum_{\forall \vect{s}_i \in Z} (Z(\vect{s}_i) - \hat{Z}(\vect{s}_i))^2
\end{equation}

TODO: put in plots and resuts

\input{sections/chapters/results_random_seed_1}

% \section{Comparing The Methods}

% \begin{table}[ht!]
% \centering
%   \begin{tabular}{ |p{6cm}||p{1cm}|p{1cm}|p{1cm}|  }
%       \hline
%       \multicolumn{4}{|c|}{Average Final Field Prediction Errors ($\sigma_{field} = 1$)} \\
%       \hline
%       Maximum Percentage Scanned    & 10\% & 20\% & 30\% \\
%       \hline
%       Zig-Zag                       & -- & -- & -- \\
%       Next Highest Value            & -- & -- & -- \\
%       $N$ Next Highest Value        & -- & -- & -- \\
%       Monte Carlo Path Planner      & -- & -- & -- \\
%       \hline
%   \end{tabular}
%   \caption{Comparing field prediction errors for varying coverage limitations on a $100 \times 100$ size field ($\sigma_{field} = 1$). Averaged over two different runs with two random seeds.}
% \end{table}

% \begin{table}[ht!]
% \centering
%   \begin{tabular}{ |p{6cm}||p{1cm}|p{1cm}|p{1cm}|  }
%       \hline
%       \multicolumn{4}{|c|}{Average Final Field Prediction Variances ($\sigma_{field} = 1$)} \\
%       \hline
%       Maximum Percentage Scanned    & 10\% & 20\% & 30\% \\
%       \hline
%       Zig-Zag                       & -- & -- & -- \\
%       Next Highest Value            & -- & -- & -- \\
%       $N$ Next Highest Value        & -- & -- & -- \\
%       Monte Carlo Path Planner      & -- & -- & -- \\
%       \hline
%   \end{tabular}
%   \caption{Comparing field prediction variances for varying coverage limitations on a $100 \times 100$ size field ($\sigma_{field} = 1$). Averaged over two different runs with two random seeds.}
% \end{table}

% \begin{table}[ht!]
% \centering
%   \begin{tabular}{ |p{6cm}||p{1cm}|p{1cm}|p{1cm}|  }
%       \hline
%       \multicolumn{4}{|c|}{Average Final Field Prediction Errors ($\sigma_{field} = 100$)} \\
%       \hline
%       Maximum Percentage Scanned    & 10\% & 20\% & 30\% \\
%       \hline
%       Zig-Zag                       & -- & -- & -- \\
%       Next Highest Value            & -- & -- & -- \\
%       $N$ Next Highest Value        & -- & -- & -- \\
%       Monte Carlo Path Planner      & -- & -- & -- \\
%       \hline
%   \end{tabular}
%   \caption{Comparing field prediction errors for varying coverage limitations on a $100 \times 100$ size field ($\sigma_{field} = 100$). Averaged over two different runs with two random seeds.}
% \end{table}

% \begin{table}[ht!]
% \centering
%   \begin{tabular}{ |p{6cm}||p{1cm}|p{1cm}|p{1cm}|  }
%       \hline
%       \multicolumn{4}{|c|}{Average Final Field Prediction Variances ($\sigma_{field} = 100$)} \\
%       \hline
%       Maximum Percentage Scanned    & 10\% & 20\% & 30\% \\
%       \hline
%       Zig-Zag                       & -- & -- & -- \\
%       Next Highest Value            & -- & -- & -- \\
%       $N$ Next Highest Value        & -- & -- & -- \\
%       Monte Carlo Path Planner      & -- & -- & -- \\
%       \hline
%   \end{tabular}
%   \caption{Comparing field prediction variances for varying coverage limitations on a $100 \times 100$ size field ($\sigma_{field} = 100$). Averaged over two different runs with two random seeds.}
% \end{table}
