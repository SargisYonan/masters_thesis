
\chapter{Introduction}
% Why is this problem important? 
% We are introducing the concept of aerial field exploration
Aeriel field exploration is a method in which information from an unknown ground field, a \textit{target field}, can be collected in order to discover traits or track the trends about the field. An exploration method could be a portable generic method for field exploration where a model of the target field phenomena is not known. Field exploration methods can be useful for tracking the health of crop soil, the size of ice glaciers, generating terrain maps, and a wide variety of scientific purposes. Furthermore, an exploration technique, versus a patrolling or tracking technique, does not require a model of the target field dynamics, as they can be learned on-the-fly. This means that a variety of fields can be explored without the knowledge of the initial state of the field.

%% Here we are going to argue why UAVs are the medium in which we should choose for our explorations
Using an Unmanned Aerial Vehicle (UAV) system, for example, an unknown field of interest can be scanned within a more reasonable time frame compared to conventional scanning techniques involving satellite and manned-airplane missions. Potentially more nuanced data can be gathered from the UAV made observations because of more desirable fields of view and more customizable sensors on-board. Using the techniques introduced, a high-quality map can be generated of a previously unknown field of interest. Satellite imagery of Earth has been used for measuring various natural phenomena in the past several decades. Estimating polar ice cap melting rates and exploring the location of an oil spills are among the class of problems solved by this technology. Currently, using a service like the US Forest Services' Moderate Resolution Imaging Spectroradiometer (MODIS) Active Fire Mapping Program, images are updated every 1 to 2 days with a fixed sensor. While this program is helpful for detecting large events with long periods of activity, the sampling rate of this service might not give an emergency response team or a scientist the required resolution and precision in gathered data at their desired rate. The resolution and frequency problem along with the cost associated with building, launching, and maintaining an orbiting Earth satellite might even make some areas of research prohibitive. The use of unmanned aerial vehicles (UAVs) have more recently been used in similar fields of study and in environmental protection. The benefits gained from using UAVs is that of more rapidly acquired data with more easily adjusted accuracy. A UAV can give more nuanced and detailed data on features of a field that are not observable from the distance or field of view of an orbiting satellite. This is because the UAV can be equipped with any compatible sensor and can be deployed from virtually anywhere to fly virtually anywhere.

% What have others done before me? How is what I'm doing different? Why is it better?
Presently, a common approach to exploring a field is to conduct a predetermined zig-zag, lawn mower pattern, or another predetermined maneuver on a target field. This task might take longer than needed to collect the required data, and could potentially ineffectively use the flight or drive time of the exploration vehicle which often has a short and limited runtime. Furthermore, scanning every point in a large unknown field is an unrealistic expectation for vehicles with limited maneuvering capabilities. This is especially a problem if the field as a whole is very large and needs to only be explored to a small degree of confidence. A scheme for minimal and high-quality scanning via effective path planning would be in the benefit of time for the user(s) of the system, and the scanning equipment as well.

Due to the nature of much of the phenomena one might be interested in scanning in an unknown field, a method that exploits the known stochastic properties of the field could be used to decrease flight time. A field that exhibits properties of \textit{geospatial autocorrelation} will be more statistically exploitable to find patterns in field states, in an effort to avoid scanning more points than needed. The Kriging Method, a popular interpolation tool, offers a prediction and a variance of prediction for points in a geospatially autocorrelated field. By exploiting the Kriging variances generated by the predictions, variance based path planning methods can be used to steer an exploration vehicle in the areas of maximal uncertainty, while traversing over other areas of low prediction confidence. The methods introduced attempt to help a user of this system explore an unknown field with a known degree of confidence that is configurable through a desired runtime.
