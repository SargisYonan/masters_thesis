
\begin{abstract}
A set of methods for the exploration of unknown semi-to-fully-ergodic fields of interest are introduced. Using observations of a single state of interest from an autonomous exploration vehicle with turn rate control, a field of interest can be learned more accurately and efficiently versus preplanned scanning techniques.

The Kriging Method, a \textit{Best Linear Unbiased Predictor} (BLUP) commonly used in the field of Geospatial Analysis, is used to exploit the statistical properties, namely the geospatial autocorrelation, of a target field. The Kriging Method predicts the state of unobserved points from a set of observed points. A prediction and confidence of prediction of the entirety of a given target field can be generated from the method. From the variances associated with the predictions made by the Kriging Method, a set of path planning methods for an autonomous exploration vehicle will be introduced for the purposes of field exploration. 

The path planners can be used to reduce the overall uncertainty of field predictions by steering a single vehicle through the field to optimally collect a good set of samples to make a field prediction from. A metric for return on investment of executing a trajectory using feedback from Kriging predictions is introduced as well. The three path planners introduced all aim to supress the overall uncertainty of a Kriging prediction of an unknown generic target field of interest in order to create a variable quality map. 

\end{abstract}
